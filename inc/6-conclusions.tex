\section{ИНТЕРПРЕТАЦИЯ РЕЗУЛЬТАТОВ, ВЫВОДЫ}

В данной работе рассматривается задача, состоящая из двух частей. На первом шаге необходимо отсортировать 2 массива по убыванию, на втором --- вычислить их скалярное произведение. В качестве алгоритма сортировки был выбран \textbf{алгоритм сортировки пузырьком}. Были реализованы две версии решения задачи --- последовательная и оптимизированная с помощью CUDA, кроме того была проведено измерение времени работы реализаций при различной вычислительной трудоёмкости. В результате чего было установлено, что при малой вычислительной трудоёмкости (при n < 8192) быстрее оказывается последовательная версия. При большой вычислительной сложности задачи (при n > 8192) время работы оптимизированной версии становится меньше времени исполнения последовательного алгоритма, что можно связать с тем, что при увеличения вычислительной сложности все меньшую часть времени исполнения занимают транзакции между GPGPU и CPU и всё большая часть времени работы уходит под параллельную часть программы.