\section{ОПИСАНИЕ ПРОГРАММНО-АППАРАТНОЙ КОНФИГУРАЦИИ ТЕСТОВОГО СТЕНДА}

В качестве тестового стенда выступала вычислительная машина, доступ к которой был предоставлен преподавателем. Краткое описание программно-аппаратной конфигурации тестового стенда приведено в Таблице \ref{tab:1}.
\begin{table}[!ht]
    \centering
    \begin{tabular}{|l|l|}
    \hline
        ОС & Ubuntu 20.04.4 LTS \\ \hline
        Число ядер & 6 \\ \hline
        Число потоков & 12 \\ \hline
        Модель процессора & Intel(R) Xeon(R) E-2136 CPU @ 3.30GHz \\ \hline
        ОЗУ & 62 Гб \\ \hline
        Вычислетельная способность девайса & 61 \\ \hline
        Имя девайса &	Quadro P2000\\ \hline
        Общий размер глобальной памяти девайса & 5290131456 байт \\ \hline
        Размер разделяемой памяти на блок девайса &	49152 байт \\ \hline
        Число регистров на один блок девайса &	65536 \\ \hline
        Размер варпа девайса  &	32 \\ \hline
        Максимальное число потоков на блок девайса &	1024\\ \hline
        Общий размер константной памяти девайса &	65536 байт \\ \hline
    \end{tabular}
    \caption{Программно-аппаратная конфигурация тестового стенда.}
    \label{tab:1}
\end{table}